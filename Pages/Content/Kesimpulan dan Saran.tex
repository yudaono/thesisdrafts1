\chapter{KESIMPULAN DAN SARAN}

\section{Kesimpulan}
Telah dibandingkan dua instrument \textit{Dynamic Light Scattering} (DLS) sederhana dengan
laser yang berbeda yaitu laser merah dan laser hijau yang memiliki hasil berkisar diantara
${533 nm}$ dan ${488 nm}$ untuk laser hijau dan ${398 nm}$ dan ${383 nm}$ untuk laser merah
dan sebagai literatur didapatkan data berupa pengukuran PSA sebesar ${463.1 nm}$
\begin{enumerate}
    \item Laser merah memiliki range persebaran data yang lebih lebar dibandingkan
    dengan laser hijau. Disisi lain pengukuran dengan laser hijau menghasilkan diameter yang
    dominan lebih kecil dibandingkan laser merah.
    \item Responsivitas dari silikon pada fotodioda OPT101 memiliki nilai lebih rendah
    pada laser hijau dibandingkan dengan laser merah, namun intensitas dari laser hijau lebih
    sehingga sensitivitas pada sensor meningkat dibandingkan dengan laser merah
\end{enumerate}


\section{Saran}
Sejumlah ide yang muncul ketika melaksanakan penelitian TA dapat menjadi bahan atau topik untuk pekerjaan selanjutnya. Hal ini dapat berupa perbaikan atau ragam lain dari apa yang telah dilakukan sepanjang penelitian. Sub bab ini menjadi sumber informasi penting bagi, utamanya mahasiswa, yang akan melakukan penelitian lanjutan.