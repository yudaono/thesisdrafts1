\chapter{KESIMPULAN DAN SARAN}

\section{Kesimpulan}
Telah dibandingkan dua instrument \textit{Dynamic Light Scattering} (DLS) sederhana dengan
laser yang berbeda yaitu laser merah dan laser hijau yang memiliki hasil berkisar diantara
${365 nm}$ untuk laser hijau dan ${398 nm}$ dan ${619 nm}$ untuk laser merah.
Sebagai referensi didapatkan data pengukuran PSA berupa Z-Average pada ${463.1 nm}$
\begin{enumerate}
    \item Laser merah memiliki range persebaran data yang lebih lebar dibandingkan
    dengan laser hijau. Disisi lain pengukuran dengan laser hijau menghasilkan diameter yang
    dominan lebih kecil dibandingkan laser merah.
    \item Responsivitas dari silikon pada fotodioda OPT101 memiliki nilai lebih rendah
    pada laser hijau dibandingkan dengan laser merah, namun intensitas dari laser hijau lebih
    sehingga sensitivitas pada sensor meningkat dibandingkan dengan laser merah
\end{enumerate}


\section{Saran}
Berdasarkan penelitian yang telah dilakukan, sampel yang dapat diukur masih terlalu
sedikit untuk memastikan akurasi pengukuran yang dilakukan oleh rangkaian yang telah
dibuat. Hal tersebut memberikan saran berupa adanya variasi sampel yang lebih banyak
namun masih dalam range pengukuran dengan menggunakan laser merah dan laser hijau
agar mendapatkan data yang beragam.