\chapter{PENDAHULUAN}
\pagenumbering{arabic}

\section{Latar Belakang}
Perkembangan teknologi dalam pengamatan nanopartikel kini sudah berkembang pesat terutama dalam tes
diagnosa untuk penyakit yang berkaitan dengan virus dimana hal tersebut sangat diminati dalam sains
dan industri. Sebagai contohnya yaitu pandemi yang melanda dunia selama dua tahun terakhir.
Varian Omicron yang tersebar hampir ke seluruh dunia dengan skala penyebaran yang tidak terprediksi
dan membahayakan kesehatan. Hal tersebut membuktikan bahwa nanopartikel memiliki sifat fisika dan
kimia yang bervariasi. Bagi peneliti, perubahan sifat dari beberapa nanopartikel dapat dimodifikasi
dengan perlakuan khusus seperti mengontrol ukuran material maupun komposisi kimiawi\cite{Silva2022}.
Modifikasi tersebut mempengaruhi pergerakan nanopartikel, saat jarak antar nanopartikel terlalu dekat
ataupun terlalu jauh maka tidak akan terjadi tumbukan antar nanopartikel.

Pada saat sebuah partikel bertumbukan dengan partikel lain yang lebih besar, partikel besar dapat
membelokan arah gerak dari partikel kecil. Hal tersebut dapat diinterpretasikan dengan cahaya yang
merupakan gelombang atau partikel yang bergerak dengan arah tertentu. Adanya partikel lain yang
tersebar di dalam sebuah sistem menyebabkan cahaya tersebut dapat bergerak lurus, dipantulkan,
ataupun dihamburkan ke arah tertentu. Hal tersebut dikuatkan dalam percobaan penghamburan cahaya
yang dilakukan John Tyndall dalam penelitian suspensi koloid\cite{Goldburg1999,Falke2019}. 

Pembelokan cahaya bergantung dari partikel yang membelokan arah cahaya tersebut. Semakin besar
dimensi yang dimiliki partikel, maka pembelokan cahaya akan semakin kecil. Adapun hubungan dari
hamburan cahaya oleh partikel dengan ukuran tertentu terhadap panjang gelombang yang mengenai
partikel tersebut adalah panjang gelombang sebanding dengan ukuran dari partikel yang ditembakan
oleh cahaya tersebut (${\lambda/d \thicksim \theta}$), dan berbanding terbalik dengan sudut hamburan
cahayanya. Hal tersebut yang menjelaskan mengenai warna yang dapat dilihat seperti atmosfer langit
yang berwarna biru, api yang berwarna merah, dan lain-lain. 

Hamburan cahaya bukan satu satunya cara untuk mengidentifikasi ukuran dari sebuah partikel. Selain
dari hamburan cahaya, terdapat beberapa metode lain untuk mengidentifikasi ukuran dari nanopartikel.
Sebagai contoh yaitu \textit{Scanning Electron Microscope} (SEM) yang menggunakan berkas elektron
untuk menggambarkan profil permukaan benda dengan menembakan permukaan benda dengan berkas elektron
berenergi tinggi. Selanjutnya, Metode \textit{Transmission Electron Microscopy} (TEM) yang
memanfaatkan prinsip kerja dari peralatan Rontgen. Terakhir yaitu metode BET yang memanfaatkan
fenomena adsorpsi molekul gas di permukaan zat padat. Ketiga metode tersebut dapat merusak atau
mengubah komposisi dari zat tersebut. Di sisi lain, pengoperasian dari alat tersebut juga cukup
mahal[Khairurrijal2009].

Pada penelitian ini digunakan pengukuran dengan menggunakan sistem Dynamic Light Scattering (DLS)
yang dilakukan dengan mengamati fluktuasi dari hamburan cahaya laser. Pada spektrum warna, cahaya
dengan warna tertentu memiliki panjang gelombang dengan jangkauan tertentu. Partikel yang memiliki
ukuran dengan jangkauan radius panjang gelombang cahaya tersebut akan memantulkan atau menghamburkan
cahaya yang datang. Pada umumnya penggunaan DLS berbasiskan laser merah yang memiliki panjang
gelombang di antara ${630 nm}$ hingga ${670 nm}$. Jika menggunakan laser hijau yang memiliki panjang
gelombang di antara ${520 nm}$ hingga ${532 nm}$, maka akan didapatkan data hasil pengukuran
partikel yang memiliki ukuran lebih kecil dibanding menggunakan laser merah\cite{Black1996}. 

Respon dari sensor yang digunakan untuk menangkap fluktuasi hamburan cahaya tersebut berperan besar
untuk mengukur ukuran dari partikel. Sensor yang digunakan dalam sistem DLS ini adalah Fotodioda
dengan tipe OPT101 dan BPW34. Kedua sensor ini memiliki karakteristik dari Silikon sebagai bahan
utama penyusun Fotodioda. Responsivitas dari silikon mempengaruhi konversi dari intensitas cahaya
yang masuk menjadi arus listrik. Responsivitas dari silikon merupakan pengukuran sensitivitas cahaya
yang datang, dan didefinisikan sebagai rasio dari Fotoelektrik ${I}$ terhadap daya dari cahaya
${P}$ pada panjang gelombang tertentu. (${R = I/P}$). Responsivitas akan berubah bergantung pada
daya dari cahaya pada panjang gelombang tertentu. Di sisi lain sensor memiliki Efisiensi Kuantum
yang mengukur efektivitas dari sebuah sensor untuk mengkonversi kuatnya foton menjadi elektron.
Hal tersebut mempengaruhi responsivitas dari silikon pada fotodioda terhadap panjang gelombang
tertentu. Oleh karena itu penelitian ini diharapkan dapat membuktikan pengaruh panjang gelombang
terhadap pengukuran partikel berbasis Dynamic Light Scattering


\section{Identifikasi Masalah}
Identifikasi masalah dari penelitian ini adalah sebagai berikut:
\begin{enumerate}
  \item Bagaimana efek dari panjang gelombang laser merah dan hijau terhadap hasil pengukuran DLS?
  \item Bagaimana responsivitas fotodioda terhadap laser merah dan hijau?
\end{enumerate}

\section{Batasan Masalah}
\begin{enumerate}
    \item Partikel yang dapat diukur bergantung dari panjang gelombang Laser Hijau
    \item Partikel yang akan diukur merupakan partikel terdispersi yang sudah diukur sebelumnya
    \item Nilai sudut hamburan yang dibuat 90 derajat dari sudut datang laser
\end{enumerate}

\section{Tujuan Penelitian}
\begin{enumerate}
  \item Mengidentifikasi hasil dari metode Dynamic Light Scattering antara laser merah dan laser hijau,
  \item Menganalisa efek dari panjang gelombang laser merah dan hijau terhadap hasil pengukuran pada Dyamic Light Scattering.
\end{enumerate}

\section{Manfaat Penelitian}
Manfaat yang diharapkan dari penelitian ini yaitu dapat menganalisa hasil pengukuran dari alat ukur
partikel berbasis Dynamic Light Scattering sederhana dengan range pengukuran berdasarkan panjang
gelombang Laser Merah dan Laser Hijau
