\chapter*{\centering KATA PENGANTAR}
\addcontentsline{toc}{chapter}{KATA PENGANTAR}

Puji syukur kepada Allah SWT, atas berkat dan karunia-Nya sehingga penulis dapat menyelesaikan
skripsi yang berjudul "Efek Laser Merah dan Laser Hijau pada Pengukuran Ukuran Partikel Berbasis
Dynamic Light Scattering" sebagai salah satu syarat mengikuti mata kuliah Skripsi pada Program
Studi Fisika Fakultas Matematika dan Ilmu Pengetahuan Alam Universitas Padjadjaran. Penelitian
dan penulisan skripsi ini dapat diselesaikan dengan bantuan, bimbingan, serta nasehat dari
berbagai pihak

Untuk itu ucapan terimakasih penulis sampaikan kepada: 

\begin{enumerate}
	\item Prof. Dr. Cammelia Panatarani M.Si selaku Kepala Departemen Fisika
Fakultas Matematika dan Ilmu Pengetahuan Alam Universitas Padjadjaran,
	\item Dr. Sahrul Hidayat, M.Si selaku Ketua Program Studi Fisika Fakultas
Matematika dan Ilmu Pengetahuan Alam Universitas Padjadjaran,
	\item \supervisorNameF ~selaku pembimbing utama dan \supervisorNameS ~selaku
pembimbing pendamping yang telah banyak membantu, mengarahkan, dan memberikan ilmu
yang sangat berharga untuk penulis selama penelitian dan penulisan skripsi ini,
	\item Noto Susanto Gultom, MSc., Ph.D dan Norman Syakir, Drs., MS., M.Sc selaku
dosen penguji yang telah memberikan banyak memberikan masukan, ilmu, dan saran yang
sangat membangun dalam menyelesaikan penulisan skripsi ini,
	\item Seluruh dosen dan civitas Program Studio Fisika Fakultas Matematika dan
Ilmu Pengetahuan Alam Universitas Padjarjaran yang telah membantu penulis selama masa
perkuliahan,
	\item Kedua Orang Tua penulis yang selalu senantiasa memberikan do'a, semangat,
dan motivasi kepada penulis selama masa perkuliahan hingga penulis menyelesaikan skripsi
ini,
	\item Albiruni Mbani Wibawa, M Naufal Ardian, M Fahmi Khoirurrijal, Nur Fajria Fitri,
Alvi Avivah Nur Azizah, Dewi Asaningsih, M Galih Prawiradilaga, M Fahmi Fauzi, Syarif Aulia
Rahman dan  yang telah membantu penulis baik secara materi maupun moral selama menyelesaikan
penelitan dan penulisan skripsi ini,
	\item Rekan-rekan Treddian 2019 yang telah memberikan banyak pembelajaran selama 
masa perkuliahan,
	\item Pihak lain yang tidak dapat disebutkan satu-persatu yang telah membantu baik secara langsung maupun tidak langsung, terima kasih atas segala dukungan yang diberikan kepada penulis.
\end{enumerate}

Dengan segala kerendahan hati, penulis menyadari bahwa skripsi ini masih memiliki banyak kekurangan dan jauh dari kesempurnaan. Oleh karena itu, kritik dan saran yang membangun sangat dibutuhkan oleh penulis. Semoga skripsi ini dapat bermanfaat bagi perkembangan ilmu dan teknologi serta bagi pihak-pihak yang membutuhkan.

\vspace{1cm}
\begin{flushright}
	Jatinangor, \dateOf\\
	\vspace{0.5cm}
	\fullName
\end{flushright}