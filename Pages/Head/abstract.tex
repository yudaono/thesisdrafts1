\chapter*{\centering \textit{ABSTRACT}}
\addcontentsline{toc}{chapter}{ABSTRACT}

\begin{spacing}{1.25}
\noindent\textit{
    The Dynamic Light Scattering-based nanoparticle measurement offers advantages such
    as fast measurement time and relatively low operating costs without damaging the
    structure of the measured sample compared to other methods like Scanning Electron
    Microscopy (SEM) and Brunauer Emmett Teller (BET). Essentially, Dynamic Light
    Scattering utilizes the scattering of light from dispersed particles. Light is
    an electromagnetic wave with varying wavelengths. When light is directed at a
    particle of a certain size, the direction of light motion can be scattered at
    a specific scattering angle. This principle is described in Rayleigh Scattering,
    where the incoming light intensity differs from the intensity after passing through
    an object, influenced by the wavelength of light and the scattering angle. Object
    size also affects the scattering angle of light; smaller particles produce smaller
    scattering angles at the same wavelength. This research aims to determine if green
    laser can detect smaller particles compared to red laser in Dynamic Light Scattering
    measurements. The setup uses affordable and readily available components like
    Arduino Uno and OPT101 photodiode. The scattering angle is limited to ${90\degree}$
    from the laser light source. Particle size is calculated from the cumulative values
    detected by the sensor, followed by fitting to obtain diffusion coefficient values.
    The Stokes Einstein equation is then employed to derive particle radius values.
    The tested ${SiO_2}$ sample yielded measurements ranging from 398nm to 619nm for
    red laser and 365nm for green laser, with a reference obtained from PSA,
    the Z-Average, at 463.1nm \\
\textbf{Keywords:} Dynamic Light Scattering, Scattering Angle, Wavelength, Light}
\end{spacing}