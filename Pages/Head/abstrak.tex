\chapter*{\centering ABSTRAK}
\addcontentsline{toc}{chapter}{ABSTRAK}

\begin{spacing}{1.25}
\noindent Pengukuran nanopartikel berbasis \textit{Dynamic Light Scattering}
memiliki keunggulan, seperti waktu pengukuran yang cepat dan biaya pengoperasian
yang relatif murah tanpa merusak struktur sampel yang diukur, jika dibandingkan
dengan metode lain seperti \textit{Scanning Electron Microscopy} (SEM) dan 
\textit{Brunauer Emmett Teller} (BET). Pada dasarnya, metode 
\textit{Dynamic Light Scattering} menggunakan hamburan cahaya dari partikel yang
terdispersi. Cahaya merupakan gelombang elektromagnetik dengan panjang gelombang
yang bervariasi. Ketika cahaya diarahkan ke sebuah partikel berukuran tertentu,
arah gerak cahaya dapat terhambur dengan sudut hamburan tertentu. Prinsip ini
dijelaskan dalam \textit{Rayleigh Scattering} di mana intensitas cahaya yang
masuk berbeda dengan intensitas setelah melewati objek, dipengaruhi oleh panjang
gelombang dan sudut hamburan cahaya. Ukuran objek juga memengaruhi sudut hamburan
cahaya; partikel yang lebih kecil menghasilkan sudut hamburan yang lebih kecil
pada panjang gelombang yang sama. Penelitian ini bertujuan membuktikan apakah
laser hijau dapat mendeteksi partikel yang lebih kecil dibandingkan dengan laser
merah dalam pengukuran \textit{Dynamic Light Scattering}. Rangkaian yang
digunakan memanfaatkan komponen yang murah dan mudah ditemukan, seperti
Arduino Uno dan fotodioda OPT101. Sudut hamburan dibatasi pada ${90\degree}$ dari
arah cahaya laser. Ukuran partikel dihitung dari kumulasi nilai yang terdeteksi
oleh sensor, kemudian dilakukan fitting untuk mendapatkan nilai koefisien difusi,
menggunakan persamaan Stokes Einstein untuk mendapatkan nilai jari-jari partikel.
Hasil pengukuran dari sampel yang diuji, yaitu ${SiO_2}$, berkisar antara 
${398 nm}$ dan ${619 nm}$ untuk laser merah, serta ${365 nm}$ untuk laser hijau.
Referensi dari karakterisasi PSA, yaitu Z-Average, adalah sebesar ${463.1 nm}$.


\noindent\textbf{Kata kunci:} \textit{Dynamic Light Scattering}, Sudut Hamburan,
Panjang Gelombang, Cahaya
\end{spacing}